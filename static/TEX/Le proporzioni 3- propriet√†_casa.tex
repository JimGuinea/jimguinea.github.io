\documentclass[14pt]{extarticle}
\usepackage[utf8]{inputenc}
\usepackage{tkz-euclide}
\usepackage{tikz} 
\usepackage{pdflscape}
\usepackage[margin=2cm]{geometry}
\usepackage{amsmath}
\usepackage{pgfplots}
\usepackage{cancel}
\usepackage{setspace}
\pagenumbering{gobble} 

\setlength\parindent{0pt}
\pgfplotsset{compat=1.16}  
\usetikzlibrary{arrows.meta}

\begin{document}
\begin{center}
    \LARGE{\textbf{Le Proporzioni - Proprietà}}
\end{center}
\vspace{1cm}
Le \textbf{proprietà delle proporzioni} sono:
\begin{itemize}
    \item Proprietà fondamentale
    \item Proprietà dell'invertire
    \item Proprietà del permutare
    \item Proprietà del comporre
    \item Proprietà dello scomporre
\end{itemize}
\textbf{Proprietà fondamentale:} in ogni proporzione il prodotto dei medi è sempre uguale al prodotto degli estremi. [Proprietà fondamentale per verificare che una proporzione sia scritta correttamente]\\
in \textbf{generale:}
\[se\quad a:b=c:d\]
\[allora \quad a\cdot d=b\cdot c\]
Esempio: 
\[4:6=2:3\]
\[4\cdot3=12\]
\[6\cdot2=12\]

\textbf{Proprietà dell'invertire:} Se in una proporzione si scambia ogni antecedente con il proprio conseguente, si ha ancora una proporzione.\\
in \textbf{generale:}
\[se\quad a:b=c:d\]
\[allora \quad b:a=d:c\]
Esempio:
\[4:16=2:8\]
se applico la proprietà:
\[16:4=8:2\]
verifico la proporzione:
\[16\cdot2=32\]
\[4\cdot8=32\]

\textbf{Proprietà del permutare:} se in una proporzione si scambiano tra loro i due medi, i due estremi o entrambi, si ha ancora una proporzione. \\
In \textbf{generale:}
\[se\quad a:b=c:d\]
\[allora\quad anche\quad a:c=b:d\quad e\quad d:b=c:a\quad e\quad d:c=b:a\]
Esempio:
\[16:18=8:9\]
applico la proprietà: 
\[16:8=18:9\]
verifico la proporzione:
\[8\cdot18=144\]
\[16\cdot9=144\]
applico la proprietà: 
\[9:18=8:16\]
verifico la proporzione:
\[18\cdot8=144\]
\[9\cdot16=144\]
applico la proprietà: 
\[9:8=18:16\]
verifico la proporzione:
\[8\cdot18=144\]
\[9\cdot16=144\]

\textbf{Proprietà del comporre:} in una proporzione la somma tra il primo e il secondo termine sta al primo (o al secondo) termine come la somma tra il terzo e il quarto sta al terzo (o al quarto) termine.\\
In \textbf{generale:}
\[se\quad a:b=c:d\]
\[allora\quad(a+b):a= (c+d):c \quad e \quad (a+b):b=(c+d):d\]
Esempio:\\
\[4:8=3:6\]
applico la proprietà:
\[(4+8):4=(3+6):3\]
\[12:4=9:3\]
verifico che la proporzione sia vera:
\[12\cdot3=36\]
\[4\cdot9=36\]
\[-\]
\[(4+8):8=(3+6):6\]
\[12:8=9:6\]
verifico che la proporzione sia vera:
\[12\cdot6=72\]
\[8\cdot9=72\]

\textbf{Proprietà dello scomporre:} in una proporzione (avente li antecedenti maggiori dei rispettivi conseguenti) la differenza tra il primo e il secondo termina sta al primo (o al secondo) termine come la differenza tra il terzo e il quarto termine sta al terzo (o al quarto) termine. \\
In \textbf{generale:}    
\[se\quad a:b=c:d\ \qquad(a>b \quad e \quad c>d)\]
\[allora\quad (a+b):a=(c+d):c\]
Esempio:
\[18:30=3:5\]
applico la proprietà:
\[(30-18):30=(5-3):5\]
\[12:30=2:5\]
verifico che la proporzione sia vera:
\[30\cdot2=60\]
\[12\cdot5=60\]
\[-\]
\[(30-18):30=(5-3):5\]
\[12:18=2:3\]
verifico che la proporzione sia vera:
\[18\cdot2=36\]
\[12\cdot3=36\]
\clearpage
\textbf{Esercizi}\\
Applicando la \textbf{proprietà fondamentale} stabilisci se le seguenti scritture sono proporzioni:\\
\begin{spacing}{2.1}
\begin{minipage}[t]{0.5\textwidth}
\(55:5=33:3\)\\
\(44:56=60:76\)\\
%\(44:55=16:20\)\\
%\(32:15=10:5\)\\
\(1,1:0,5=2,8:0,6\)\\
\(2,4:3,2=1,5:2\)\\
\end{minipage}
\begin{minipage}[t]{0.5\textwidth}
\(\dfrac{3}{4}:\dfrac{1}{2}=\dfrac{4}{5}:\dfrac{8}{15}\)\\
\(\dfrac{3}{5}:\dfrac{4}{3}=\dfrac{1}{8}:\dfrac{5}{18}\)\\
%\(\dfrac{3}{10}:\dfrac{10}{3}=\dfrac{3}{4}:\dfrac{4}{3}\)\\
%\(\dfrac{1}{2}:\dfrac{4}{3}=\dfrac{7}{8}:\dfrac{3}{7}\)\\
\(\dfrac{4}{7}:\dfrac{1}{5}=\dfrac{3}{7}:\dfrac{3}{20}\)\\
\end{minipage}
\end{spacing}
Applica la \textbf{proprietà dell'invertire} alle seguenti proporzioni e verifica che sia proporzioni corrette (usa la proprietà fondamentale):\\
\begin{minipage}[t]{0.5\textwidth}
\(18:72=26:104\)\\
\(150:25=234:39\)\\
\end{minipage}
\begin{minipage}[t]{0.5\textwidth}
\(1,6:5,4=2,2:7,425\)\\
\(7,75:6,2=0,5:0,4\)\\
\end{minipage}
Applica la \textbf{proprietà del permutare} in tutti i modi possibili alle seguenti proporzioni e verifica che sia proporzioni corrette (usa la proprietà fondamentale):\\
\(25:35=5:7\)\\
\(50:30=25:15\)\\

Applica la \textbf{proprietà del comporre} alle seguenti proporzioni e verifica che sia proporzioni corrette (usa la proprietà fondamentale):\\
\(64:16=24:6\)\\
\(54:72=18:24\)\\

Applica la \textbf{proprietà dello scomporre} alle seguenti proporzioni e verifica che sia proporzioni corrette (usa la proprietà fondamentale):\\
\(39:3=78:6\)\\
\(16:11=80:55\)\\




\end{document}
