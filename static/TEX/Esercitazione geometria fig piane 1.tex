\documentclass[14pt]{extarticle}
\usepackage[utf8]{inputenc}
\usepackage{tkz-euclide}
\usepackage{tikz} 
\usepackage{pdflscape}
\usepackage[margin=2cm]{geometry}
\usepackage{amsmath}

\begin{document}
\begin{center}
    \LARGE{\textbf{Esercitazione 1 geometria - 23/09/2022 - 3B}}
\end{center}
\vspace{1cm}
\textbf{Ricopia sul quaderno le figure (col righello!!!) e i dati, trova le informazioni richieste e scrivi sotto ogni figura il nome.}\vspace{1.5cm}\\

%----- FINE CONSEGNA -----

%parallelogramma
\begin{minipage}{0.25\linewidth}
\textbf{1)}
\begin{tikzpicture}[scale=1.1]
    %initialisation
    \tkzInit[xmin=0,xmax=4,ymin=0,ymax=2] 
    \tkzClip[space=.5] 
    %definitions
    \tkzDefPoint(0,0){A} 
    \tkzDefPoint(3,0){B} 
    \tkzDefPoint(4,2){C} 
    \tkzDefPoint(1,0){H}
    \tkzDefPointWith[colinear= at C](B,A) \tkzGetPoint{D}
    %drawing
    \tkzDrawPolygon(A,B,C,D)
    %\tkzDrawSegments[blue,dashed](A,C B,D)
    %label
    \tkzLabelPoints[below](A,B,H)
    \tkzLabelPoints[above](C,D)
    \tkzDrawSegments[style=dashed](D,H)
    %\tkzLabelSegment[above,pos=.7,sloped](A,C){$x+y$}
    %\tkzLabelSegment[above,pos=.7,sloped](B,D){$x-y$}
\end{tikzpicture}
\end{minipage}
\hspace{0.7cm}
\begin{minipage}{0.25\linewidth}
$\overline{AB}=5cm$\\
$\overline{BC}=3cm$\\
$2p=?$
\end{minipage}
% triangolo isoscele
\begin{minipage}{0.25\textwidth}
\textbf{2)}
\begin{tikzpicture}[rotate=-30]
    \tkzDefPoint(2,3){A}
    \tkzDefShiftPoint[A](0:4){B}
    \tkzDefShiftPoint[A](30:4){C}
    \tkzDefMidPoint(B,C)
    \tkzGetPoint{H}
    \tkzDrawSegments(A,B B,C C,A)
    \tkzMarkSegments[mark=|](A,B A,C)
    \tkzDrawPoints(A,B,C,H)
    \tkzLabelPoints[right](B,C, H)
    \tkzLabelPoints[above left](A)
    \tkzDrawSegments[style=dashed](A,H)
\end{tikzpicture}
\end{minipage}
\hspace{1cm}
\begin{minipage}{0.25\textwidth}
$\overline{AB}=4cm$\\
$\overline{BC}=2cm$\\
$A=3.8cm^2$\\
$altezza=?$\\
$2p=?$
\end{minipage}
\vspace{1.5cm}

%triangolo equilatero
\begin{minipage}{0.25\linewidth}
\textbf{3)}
\begin{tikzpicture}[scale=1]
    \tkzDefPoint(2,3){A}
    \tkzDefShiftPoint[A](30:3){B}
    \tkzDefShiftPoint[A](-30:3){C}
    \tkzDefMidPoint(B,C)
    \tkzGetPoint{H}
    \tkzDrawPolygon(A,B,C)
    \tkzDrawPoints(A,B,C,H)
    \tkzLabelPoints[right](B,C, H)
    \tkzLabelPoints[left](A)
    %\tkzMarkSegments[mark=|](A,B A,C B,C)
    \tkzDrawSegments[style=dashed](A,H)
\end{tikzpicture}
\end{minipage}
\begin{minipage}{0.25\linewidth}
    $\overline{AB}=5cm$\\
    $altezza=?$\\
    $2p=?$
\end{minipage}
\begin{minipage}{0.25\linewidth}
\textbf{4)}
%quadrato
\hspace{0.5cm}
\begin{tikzpicture}[scale=1, rotate=20]
    \tkzDefPoints{0/0/A,2/0/B,2/2/C,0/2/D}
    \tkzDrawPolygon(A,...,D)
    \tkzDrawPoints(A,...,D)
    \tkzLabelPoints[below](A,B)
    \tkzLabelPoints[above](C,D)
\end{tikzpicture}
\end{minipage}
\begin{minipage}{0.25\linewidth}
    $\overline{AB}=7cm$\\
    $area=?$\\
    $2p=?$   
\end{minipage}

\vspace{1.5cm}
\begin{minipage}{0.25\linewidth}
\textbf{5)}
%trapezio isoscele
\begin{tikzpicture}[scale=1, rotate=-200]
    \tkzDefPoints{0/0/A,4/0/B,3/2/C,1/2/D}
    \tkzDrawPolygon(A,...,D)
    \tkzDrawPoints(A,...,D)
    \tkzDefMidPoint(B,C)
    \tkzDefPointBy[projection=onto B--A](C) \tkzGetPoint{H} %ALTEZZA
    \tkzDrawSegments[style=dashed](C,H)
    \tkzLabelPoints[above right](A,B,H)
    \tkzLabelPoints[below left](C,D)
\end{tikzpicture}
\end{minipage}
\hspace{0.5cm}
\begin{minipage}{0.25\linewidth}
    $altezza=3cm$\\
    $\overline{AB}=10cm$\\
    $\overline{CD}=6cm$\\
    $\overline{CB}=4cm$\\
    $area=?$\\
    $2p=?$   
\end{minipage}
\begin{minipage}{0.25\linewidth}
\textbf{6)}
%trapezio isoscele 2
\begin{tikzpicture}[scale=1, rotate=-50]
    \tkzDefPoints{0/0/A,4/0/B,3/2/C,0/2/D}
    \tkzDrawPolygon(A,...,D)
    \tkzDrawPoints(A,...,D)
    \tkzDefMidPoint(B,C)
    \tkzDefPointBy[projection=onto B--A](C) \tkzGetPoint{H} %ALTEZZA
    \tkzDrawSegments[style=dashed](C,H)
    \tkzLabelPoints[above left](A,D)
    \tkzLabelPoints[right](C,B)
    \tkzLabelPoints[left](H)
\end{tikzpicture}
\end{minipage}
\hspace{0.5cm}
\begin{minipage}{0.25\linewidth}
    $\overline{AB}=12cm$\\
    $\overline{CD}=8cm$\\
    $area=48cm^2$\\
    $\overline{CB}=\frac{3}{2}\overline{DA}$\\
    $altezza=?$\\
    $\overline{Bh}=?$\\
    $2p=?$   
\end{minipage}

% --------- SOLUZIONI

\newpage
\begin{center}\textbf{Soluzioni}\\\end{center}
\textbf{Es 1}:\\
La figura è un parallelogramma, per calcolare il perimetro è sufficiente sommare i 4 lati secondo la formula generale:
$$\overline{AB}+\overline{BC}+\overline{CD}+\overline{DA}=2p$$
Visto che nel parallelogramma i lati opposti sono uguali si può scrivere in maniera più sintetica:
$$2\cdot\overline{AB}+2\cdot\overline{BC}=2p$$
E sostituendo ai segmenti i valori riportati nei dati si ha: $$2\cdot5cm+2\cdot3cm=16cm$$
%  es 2
\textbf{Es 2:}\\
La figura è un triangolo isoscele, ovvero ha due lati (\(\overline{AB}\) e \(\overline{AC}\)) congruenti (di uguale lunghezza).\\
Per trovare il perimetro si applica la formula:
$$\overline{AB}+\overline{AC}+\overline{BC}=2p$$
Oppure in maniera più sintetica:
$$2\cdot\overline{AB}+\overline{BC}=2p$$
E sostituendo ai segmenti i valori riportati nei dati si ha: $$ 2\cdot4cm +2cm=10cm$$
Per quanto riguarda il calcolo della \textit{altezza} si può o passare attraverso il \textit{Teorema di Pitagora} [che a questo punto dell'anno non abbiamo ancora ripassato, ma visto che è un argomento che avete affrontato lo scorso anno chi vuole può utilizzarlo] oppure si può ricavare dalla formula dell'area conoscendo l'altezza:
$$A_{triangolo}=\frac{base\cdot altezza}{2}$$
$$altezza=\frac{2\cdot A}{base}$$
quindi nel nostro caso:
$$altezza=\frac{2\cdot 3,8cm^2}{2cm}=3,8cm$$
\textbf{Es 3}\\
La figura è un triangolo equilatero (tutti e 3 i lati uguali (e di conseguenza anche gli angoli) ). Il perimetro si trova sommando i 3 lati:\\ $$\overline{AB}+\overline{BC}+\overline{CD}=2p$$
$$5cm+5cm+5cm=15cm$$
oppure:
$$\overline{AB}\cdot3=2p$$
$$5cm\cdot3=15cm$$
Per trovare l'altezza si può procedere in due modi: o utilizzando il \textit{Teorema di Pitagora} [che a questo punto dell'anno non abbiamo ancora ripassato] oppure utilizzando la \textit{Formula di Erone} per trovare l'area e poi da quest'ultima ricavare l'altezza.
Formula di Erone: $A_{triangolo}=\sqrt{p\cdot(p-a)\cdot (p-b)\cdot(p-c)}$ dove $p$ è il semiperimetro (metà del perimetro), $a\,b\,c$ sono le lunghezze dei lati del triangolo. Nel nostro caso diventa:
$$A_{triangolo}=\sqrt{p\cdot(p-\overline{AB})\cdot(p-\overline{BC})\cdot(p-\overline{CA})}$$
$$\sqrt{7,5cm\cdot(7,5cm-5cm)\cdot(7,5cm-5cm)\cdot(7,5cm -5cm)}=10,82cm^2$$
Da qui applicando la formula inversa per l'altezza ricavata nell'esercizio 2 si ha:
$$altezza=\frac{2\cdot 10,82cm^2}{5cm}=4,33cm$$
\\
\textbf{Es 4:}
\\La figura è un quadrato, per trovare il perimetro si possono sommare i 4 lati:
$$2p=\overline{AB}+\overline{BC}+\overline{CD}+\overline{DA}$$
$$7cm+7cm+7cm+7cm=28cm$$
Oppure in maniera più sintetica:
$$2p=\overline{AB}\cdot4=7cm\cdot 4= 28cm$$
Per quanto riguarda l'area invece si applica la formula:
$$A_{quadrato}=lato\cdot lato=lato^2=7cm\cdot7cm=28cm^2$$
\newpage
\textbf{Es 5:}\\
La figura è un trapezio isoscele (lati obliqui congruenti), per trovare il perimetro si sommano i lati (come per il quadrato e il rettangolo):
$$2p=10cm+4cm+6cm+4cm=10cm+2\cdot4cm+6cm=24cm$$
Per calcolare l'area bisogna applicare la formula:
$$A_{trapezio}=\frac{(B+b)\cdot h}{2}$$
Dove $B$ è la base maggiore e $b$ la base minore.
$$A_{trapezio}=\frac{(\overline{AB}+\overline{CD})\cdot \overline{CH}}{2}=\frac{(10cm+6cm)\cdot 3cm}{2}=24cm^2$$
\\\textbf{Es 6:}\\
La figura è un trapezio rettangolo (angolo retto tra base e lato). Per calcolare il perimetro è necessario conoscere, oltre ai dati forniti dal problema, anche il lato $\overline{DA}$ (che è anche congruente all'altezza) e il lato $\overline{CB}$ (che però dipende da $\overline{DA}$) quindi utilizzando l'area si può ricavare l'altezza e quindi il lato $\overline{DA}$:
$$A_{trapezio}=\frac{(B+b)\cdot h}{2}$$
$$h=\frac{A\cdot2}{(B+b)}=\frac{48cm^2 \cdot 2}{(12cm+8cm)}=4,8cm$$
Trovata l'altezza e quindi il lato $\overline{DA}$ si può calcolare il lato $\overline{CB}$:
$$\overline{CB}=\frac{3}{2}\overline{DA}$$
$$\overline{CB}=\frac{3}{2}\cdot4,8cm=\frac{3\cdot 4,8cm}{2}=7,2cm$$
Trovati tutti i lati si possono calcolare perimetro e area come nell'esercizio 5.
\end{document}

% Fare scritte più grosse, il probelma del trapezio rettangolo ha un problema sulle misure delle due basi, la parte tra il lato obliquo e h risulta troppo piccola rispetto alla base minore


