\documentclass[14pt]{extarticle}
\usepackage[utf8]{inputenc}
\usepackage{tkz-euclide}
\usepackage{tikz} 
\usepackage{pdflscape}
\usepackage[margin=2cm]{geometry}
\usepackage{amsmath}
\usepackage{tabularx}
\usepackage{xcolor}
\usepackage{setspace}
\setlength\parindent{0pt}

%\def\tabularxcolumn#1{m{#1}} %allinea testo al centro

\begin{document}
\begin{table}
    %\caption{Operazioni tra numeri frazionari}
    \begin{tabular}{|m{0.5\linewidth}|m{0.5\linewidth}|}
        \hline
        \multicolumn{2}{|c|}{\large{\textbf{Proprietà delle radici}}}\\
        \hline
        \multicolumn{2}{|p{0.9\textwidth}|}{\[\sqrt[b]{a}=c \hspace{1cm} c^b=a   \hspace{1cm} \sqrt[2]{9}=3 \hspace{1cm} 3^2=9 \]}\\
        \hline
        \hline
        \textbf{Teoria} & \textbf{Esempio }\\
        \hline
        \( \mathbf{\sqrt[b]{a\cdot c}=\sqrt[b]{a}\cdot\sqrt[b]{c}}\) & \[\sqrt[2]{3\cdot 2}=\sqrt[2]{3}\cdot\sqrt[2]{2}\]\\
        \hline
        \(\mathbf{\sqrt[b]{a:c}=\sqrt[b]{a}:\sqrt[b]{c}}\) & \[\sqrt[2]{3: 2}=\sqrt[2]{3}:\sqrt[2]{2}\]\\
        \hline
        \(\mathbf{(\sqrt[b]{a})^c=\sqrt[b]{a^c}}\) & \[(\sqrt[2]{3})^4=\sqrt[2]{3^4}\]\\
        \hline
        \(\sqrt[c]{\sqrt[b]{a}}=\sqrt[c\cdot b]{a}\) & \[\sqrt[2]{\sqrt[3]{4}}=\sqrt[2\cdot 3]{4}=\sqrt[6]{4}\]\\
        \hline
        \(a\cdot\sqrt[b]{c}=\sqrt[b]{a^b\cdot c}\)& \[2\cdot\sqrt[3]{4}=\sqrt[3]{2^3\cdot 4}\] \\
        \hline
        \hline
         \end{tabular}
\end{table}
         
\clearpage


\textbf{Applica le proprietà delle radici per trovare il valore delle seguenti radici, semplifica il più possibile il radicando:}\\
Esempio:\[(\sqrt[2]{3})^4=\sqrt[2]{3^4}=\sqrt[2]{3^2\cdot3^2}=\sqrt[2]{3^2}\cdot\sqrt[2]{3^2}=3\cdot3=9\]
\begin{spacing}{3}
%\vspace{.2cm}
%proprietà 1 e 4
\begin{minipage}[t]{0.5\textwidth}
\(\sqrt[2]{3\cdot4}=\)\\
\(\sqrt[2]{5}\cdot\sqrt[2]{7}=\)\\
\(\sqrt[2]{7:16}=\)\\
\(\dfrac{2\cdot\sqrt[3]{4}}{\sqrt[3]{32}}=\)\\ 
\(\dfrac{\sqrt[3]{2^3\cdot 4}}{\sqrt[2]{2}}=\)\\
\(\dfrac{\sqrt[2]{3^2\cdot 5}}{\sqrt[2]{81}}=\)\\
\(\dfrac{2^4}{\sqrt[4]{8^4\cdot 16}}=\)\\
\(\dfrac{\sqrt[2]{9\cdot 25}}{2+\sqrt[2]{9}}=\)\\
\end{minipage}
\begin{minipage}[t]{0.5\textwidth}
\(2\cdot\sqrt[2]{7}=\)\\
\(\sqrt[2]{15\cdot9}=\)\\
\(\sqrt[2]{108}=\)\\ 
\(\sqrt[3]{40}=\)\\
\(\sqrt[4]{112}=\)\\
\(\dfrac{(\sqrt[2]{2})^3}{\sqrt[2]{16}}=\)\\
\(\dfrac{(\sqrt[2]{3})^4}{\sqrt[2]{63}}=\)\\
\(\dfrac{\sqrt[2]{6}\cdot\sqrt[2]{2}}{\sqrt[2]{16}}=\)
\end{minipage}
\end{spacing}
\end{document}


