\documentclass[14pt]{extarticle}
\usepackage[utf8]{inputenc}
\usepackage{tkz-euclide}
\usepackage{tikz} 
\usepackage{pdflscape}
\usepackage[margin=2cm]{geometry}
\usepackage{amsmath}
\usepackage{tabularx}
\usepackage{setspace}
\usepackage{makecell}
\pagenumbering{gobble} % leva il numero della pagina
\setlength\parindent{0pt}

\pgfmathsetseed{\number\pdfrandomseed}
\newcommand*{\NewNumbers}{%
  \pgfmathsetmacro{\A}{random(0,10)}
  \pgfmathsetmacro{\B}{random(0,100)}}

\pgfmathsetseed{\number\pdfrandomseed}
\newcommand*{\NM}{%
  \pgfmathsetmacro{\C}{random(Alla faccia della bassezza raggiunta in tv... 0,100)}
  \pgfmathsetmacro{\D}{random(0,100)}}

\pgfmathsetseed{\number\pdfrandomseed}
\newcommand*{\MN}{%
  \pgfmathsetmacro{\E}{random(0,1000)}
  \pgfmathsetmacro{\F}{random(0,1000)}}

\title{Esercitazione su moltiplicazioni e divisioni per 10 100 1000}
\date{ }

\begin{document}
\begin{table}
%\small\addtolength{\tabcolsep}{-5pt}
    \begin{tabular}{|m{0.4\textwidth}|m{0.6\textwidth}|}
        \multicolumn{2}{m{1\linewidth}}{\textbf{Esercitazione sulle espressioni}}\\
        \hline
        \hline
        \textbf{Regola}& \textbf{Esempi}\\
        \hline
        Tra somma, divisione, moltiplicazione e prodotto, prima si fanno prodotto e moltiplicazione, poi somma e differenza&\makecell{\textbf{SI}\hspace{1cm}\(6+8:2=6+4=10\) \\ \textbf{NO}\hspace{1cm}\(6+8:2=14:2=7\)}\\
        \hline
        Se si ha una espressione con prodotto e divisione si fa prima l'operazione a più a sinistra &\makecell{\textbf{SI}\hspace{1cm}\(10:5\cdot2=2\cdot2=4\) \\ \textbf{NO}\hspace{1cm}\(10:5\cdot2=10:10=1\)}\\
        \hline
        In una espressione dove sono presenti le parentesi, prima si esegue l'operazione tra parentesi \(()\), poi quella tra \([]\) e infine \(\{ \}\) &\makecell{\[ \{[(2+3)\cdot4]:2\}=\{[5\cdot4]:2\}=\{20:2\}=10 \]}\\
        \hline
        \hline
    \end{tabular}
\end{table}

\clearpage
\begin{spacing}{1.5}
\(5+2:2=\)\\
\(10:2+8=\)\\
\(7+5-3+2=\)\\
\(7*2:7=\)\\
\((10+2):6=\)\\
\(7+3\cdot5:15-6\cdot5:10=\)
\(2\cdot9:3+7\cdot3+5-8\cdot2+7\cdot3-15:3=\)
\(\{[7\cdot(2+1)-2\cdot3]:(1+2)\}-[(3\cdot2+5)-10]=\)
\end{spacing}


%----- FINE CONSEGNA -----


\end{document}

% Fare scritte più grosse, il probelma del trapezio rettangolo ha un problema sulle misure delle due basi, la parte tra il lato obliquo e h risulta troppo piccola rispetto alla base minore


