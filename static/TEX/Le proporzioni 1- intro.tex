\documentclass[14pt]{extarticle}
\usepackage[utf8]{inputenc}
\usepackage{tkz-euclide}
\usepackage{tikz} 
\usepackage{pdflscape}
\usepackage[margin=2cm]{geometry}
\usepackage{amsmath}
\usepackage{pgfplots}
\usepackage{cancel}
\usepackage{setspace}

\setlength\parindent{0pt}
\pgfplotsset{compat=1.16}  
\usetikzlibrary{arrows.meta}

\begin{document}
\begin{center}
    \LARGE{\textbf{Le Proporzioni - Introduzione}}
\end{center}
\vspace{1cm}
\textbf{Definizione:} \textit{La proporzione è una \textbf{uguaglianza} tra \textbf{due rapporti}}\\

I quattro numeri di una proporzione si chiamano \textbf{termini della proporzione} e possono essere chiamati in 2 modi diversi:\\
in un caso il primo e il terzo si dicono \textbf{antecedenti}, il secondo e il quarto \textbf{conseguenti}.\\
In un altro caso i primo e il quarto si chiamano \textbf{estremi}, il secondo e terzo si chiamano \textbf{medi}.\\
\[a:b=c:d\]
\[\textcolor{red}{a}:\textcolor{blue}{b}=\textcolor{red}{c}:\textcolor{blue}{d}\]
\(\textcolor{red}{a}\) e \(\textcolor{red}{c}\) sono gli \textcolor{red}{antecedenti} mentre \(\textcolor{blue}{b}\) e \(\textcolor{blue}{d}\) sono i \textcolor{blue}{conseguenti}.
\[\textcolor{red}{a}:\textcolor{blue}{b}=\textcolor{blue}{c}:\textcolor{red}{d}\]
In questo caso invece \(\textcolor{red}{a}\) e \(\textcolor{red}{d}\) sono gli \textcolor{red}{estremi} e \(\textcolor{blue}{b}\) e \(\textcolor{blue}{c}\) sono i \textcolor{blue}{medi}.

Questo diverso modo di chiamare gli stessi numeri è utile nella applicazione delle diverse proprietà delle proporzioni.\\
La proporzione  
\[a:b=c:d\]
si legge: \\
\begin{center}
    \textit{"a sta a b come c sta a d"}
\end{center}
Quindi \(2:4=5:10\) si legge \textit{"2 sta a 4 come 5 sta a 10"}.\\
Le proporzioni, per essere scritte correttamente (da un punto di vista matematico), devono essere \textbf{realmente} una uguaglianza tra due rapporti, ovvero i due rapporti devono essere dello \textbf{stesso valore}. Per esempio tra le due proporzioni
\[3:6=7:14\]
\[3:6=5:15\]
solo una è corretta. \\
Osservando la prima si nota che i due rapporti \(3:6\) (o anche \(\frac{3}{6}\)) e \(7:14\) (o \(\frac{7}{14}\)) hanno lo \textbf{stesso valore}, ovvero \(0,5\) (o \(\frac{1}{2}\)). Nella seconda proporzione invece i due rapporti hanno \textbf{valori diversi}: \(3:6=\frac{1}{2}\), \(5:15=\frac{1}{3}\). \textbf{Solo la prima proporzione è corretta!}\\

\fbox{\begin{minipage}{1\textwidth}\textbf{ATTENZIONE:} facendo gli esercizi può capitare di dover fare più passaggi e quindi scrivere più volte di seguito la stessa proporzione, è \textbf{estremamente sbagliato} scrivere i diversi passaggi di una proporzione con il simbolo \textbf{\(=\)} tra un passaggio e l'altro.
%mettere esempio di passaggi successivi tra proporzioni
\end{minipage}}\\

\large{\textbf{Esercizi:}}\\
1)Per ogni proporzione indica quali numeri sono i medi, gli estremi, gli antecedenti e i conseguenti.\\

\(25:8=21:3\)\\
\(8:32=25:1000\)\\
\(3:4=24:32\)\\
\(10:5=24:12\)\\

2)Verifica se le seguenti uguaglianze sono proporzioni.\\

\begin{minipage}[t]{0.5\textwidth}
\(3:6=7:14\)\\
\(6:3=7:14\)\\
\(8:2=16:4\)\\
\(19:8=5:2\)\\
\(19:95=7:35\)\\
\end{minipage}
\begin{minipage}[t]{0.5\textwidth}
\begin{spacing}{2.1}
\(\dfrac{7}{4}:\dfrac{2}{5}=\dfrac{10}{3}:\dfrac{16}{21}\)\\
\(\dfrac{11}{6}:\dfrac{5}{12}=\dfrac{8}{15}:\dfrac{4}{33}\)\\
\(\dfrac{16}{15}:\dfrac{4}{9}=\dfrac{12}{5}:1\)\\
\end{spacing}
\end{minipage}


\end{document}
